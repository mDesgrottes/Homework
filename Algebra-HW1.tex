\documentclass{article}
\usepackage{amsmath}
\renewcommand{\baselinestretch}{1.05}
\newcommand{\norm}[1]{\left\lVert#1\right\rVert}
\usepackage{amsmath,amsthm,verbatim,amssymb,amsfonts,amscd, graphicx}
\usepackage[T1]{fontenc}
\usepackage{bigfoot} % to allow verbatim in footnote
\usepackage[numbered,framed]{matlab-prettifier}
\usepackage{filecontents}
\usepackage{graphics}
\topmargin0.0cm
\headheight0.0cm
\headsep0.0cm
\oddsidemargin0.0cm
\textheight23.0cm
\textwidth16.5cm
\footskip1.0cm
\theoremstyle{plain}
\newtheorem{theorem}{Theorem}
\newtheorem{corollary}{Corollary}
\newtheorem{lemma}{Lemma}
\newtheorem{proposition}{Proposition}
\newtheorem*{surfacecor}{Corollary 1}
\newtheorem{conjecture}{Conjecture} 
\newtheorem{question}{Question} 
\theoremstyle{definition}
\newtheorem{definition}{Definition}
\title{Matrix Theory}
\author{Mike Desgrottes}
\date{August 2020}

\begin{document}
\maketitle

\section{}
\begin{theorem}
Let V be a real vector space. Let $f: V \to [-2,2]$ be a function satisfying $$ f(au + bv) \geq \min{f(u), f(v)} $$ for all real numbers a, and b and all $u,v \in V$. Prove that $f(0) \geq f(v)$ for all $v \in V$. Also if $f(0) \geq h \geq 0$, then prove that $W = \{v \in V: f(v) \geq h \}$ is a subspace. 
 \end{theorem}
\begin{proof}
 $$f(0) = f(v - v) \geq f(v) $$ for all $v \in V$. 

 We proceed to prove that W is a subspace. First, $0 \in W$ because $f(0) \geq h$. Let a and b be real numbers and $u,v \in V$, then $f(au + bv) \geq \min{f(u),f(v)} \geq h$. Therefore, $af(u) + bf(v) \in W$. W is closed under linear combination. W contains inverse because $f(-v) \geq f(v) \implies -v \in W$. W is a subspace. 
\end{proof}
 
 \begin{theorem}
 	Let $W_{1},W_{2}$ be subspaces of a vector space V. Prove that $W_{1} \bigcup W_{2}$ is a subspace if and only if $W_{1} \subset W_{2}$ or $W_{2} \subset W_{1}$.
 \end{theorem}

\begin{proof}
	We note that $W = (W_{1}/W_{2} \bigcup W_{2}/W_{1}) \bigcup W_{1} \bigcap W_{2}$. Let $u \in W_{1}/W_{2}$ and $v \in W_{2}/W_{1}$, we begin by looking at an arbitrary linear combination of u and v. 

	If $c_{1}u + c_{2}v \in W_{1} \bigcap W_{2}$, then it implies that $c_{1}u + c_{2}v - c_{1}u \in W_{1} $ which is a contradiction since $u \not \in W_{1}$.  

	Without loss of generality, let $c_{1}u + c_{2}v \in W_{1}/W_{2}$, then either $c_{1}u + c_{2}v - c_{1}u \in W_{1}/W_{2}$ or $c_{1}u + c_{2}v - c_{1}u \in W_{1} \bigcap W_{2}$. The former cannot happen due to the fact that $v \in W_{2}/W_{1}$. $c_{1}u + c_{2}v - c_{1}u \in W_{1} \implies c_{2}v \in W_{1}$ which is a contradiction. 

	We have shown that if $W$ is a subspace, then $W_{2} \subset W_{1}$ or $W_{1} \subset W_{2}$.
	
	WLOG let $W_{1} \subset W_{2}$, then we have to show that $W_{1} \bigcup W_{2}$ is a subspace. it's trivial due to the fact that $W_{1} \bigcup W_{2} = W_{2}$ which is itself a subspace. 
\end{proof}

\begin{theorem}
Let W be a subspace of a finite dimensional vector space V. Prove that there exists a subspace U such that U is the compliment of W.
\end{theorem}

\begin{proof}
We have to show that there exists a subspace U such that 


(1) $W + U = V$

(2) $W \bigcap U = \emptyset$

Let $B = \{v_{1},....,v_{k}\}$ be a basis for V. $k < \infty$ since V is finite dimensional. Let $ 1 \leq n \leq k$, then if $B_{1} = \{ v_{1}, ..., v_{n}\} $ is a basis for W. We choose U to be the subspace generated by the span of $B_{2} = \{ v_{n + 1},....,v_{k}\}$. First, if $x \in V$, then $x = \sum_{1}^{k} a_{i}v_{i} = \sum_{1}^{n} a_{i}v_{i} + \sum_{i = n + 1}^{k} a_{i}v_{i}$. There can be no vectors in the intersection of the subspaces because it would contradict the fact that both subspaces are linearly independent.
\end{proof}
\begin{theorem}
Let $W_{1}$ and $W_{2}$ be two finite-dimensional subspaces of the vector space V. Prove that $W_{1}$ and $W_{2}$ are independent if and only if $\dim( W_{1} + W_{2}) = \dim W_{1} + \dim W_{2}$
\end{theorem}
\begin{proof}
Suppose that $W_{1}$ and $W_{2}$ are independent subspaces, and $\{v_{1},...,v_{k}\}$ and $\{ w_{1},...,w_{m}\}$ be their respective bases. then we will look at the subspace $W_{1} + W_{2}$. If we pick a vector $v \in W_{1} + W_{2}$, then $v = \sum_{i = 1}^{k} v_{i} + \sum_{i = 1}^{m} w_{i}$. The set $\{ v_{i},...,v_{k},w_{1},...,w_{m}\}$ is a spanning set. Now, it suffices to show that  the union of the two bases form a linearly independent set. Independent subspaces are disjoint. the vectors in the basis of $W_{1}$ and $W_{2}$ are linearly independent. Thus, the set $\{ v_{i},...,v_{k},w_{1},...,w_{m}\}$ is a basis for the subspace $W_{1} + W_{2}$. 

Suppose that $\dim(W_{1} + W_{2}) = \dim W_{1} + \dim W_{2}$. If $\dim W_{1} = k$ and $\dim W_{2} = m$, then a basis for $W_{1} + W_{2}$ will have $m + k$ linearly independent vectors. To show that $W_{1}$ and $W_{2}$ are independent, we show that $W_{1} \bigcap W_{2} = \emptyset$. To see this, suppose WLOG that $v_{1} \in W_{2}$, then this contradict the fact that the dimension of $W_{2} + W_{1}$ is $m + k$ because the set $\{ v_{2},...,v_{k}, w_{1},...,w_{m}\}$ would also be a basis for $W_{1} + W_{2}.$
\end{proof}

\begin{theorem}
Let V be a vector space with basis $\{v_{1},...,v_{n}\}$. Is $\{ w_{1},...,w_{n} \}$ necessarily a basis for V with $w_{i} = \sum_{j = 1}^{i} v_{j}$
\end{theorem}

\begin{proof}
We see that $\sum_{i = 1}^{n} a_{i}w_{i} = \sum_{i = 1}^{n} a_{i}\sum_{j = 1}^{i} v_{j}$ by swapping the order of summation we arrive at $\sum_{j = 1}^{n} v_{j}\sum_{i = j}^{n} a_{i} = 0$. This implies that $\sum_{i = j}^{n} a_{i} = 0$ for all $j \in [1,..n]$. Upon closer inspection, we see this implies that $a_{i} = 0$ for all i. This is because the sum in question is $a_{n}v_{n} + (a_{n} + a_{n - 1})v_{n - 1} + ... + (a_{n} + ... + a_{1})v_{1}$. it implies that all of the $a_{i} = 0$
\end{proof}

\begin{theorem}
(i) Let $W = \{ p \in \mathbb{P}_{4}: \int_{-1}^{1} p(t)dt = 0\}$. Find a basis for W.

(ii)Extend the basis in (i) to a basis of $\mathbb{P}_{4}$.

(iii) Find a subspace U such that $\mathbb{P}_{4} = W + U$.
\end{theorem}

\begin{proof}
	Let $p(t) = at^{4} + bt^{3} + ct^{2} + dt + e$, then $\rho(p) = \int_{-1}^{1} p(t) dt$. $ \rho(p) = 0 \implies e = \frac
	{-a}{5} - \frac{c}{3}$. A basis for W is $\{ 1, x, x^{3}, ax^{4} + cx^{2} - (\frac{a}{5} + \frac{c}{3})\}$.

	(iii) The subspace $U = \{p \in \mathbb{P}_{4}: \rho(p) \not = 0\}$.
\end{proof}
\begin{theorem}
Let V be a vector space of dimension n, and W be a subspace of V of dimension n - 1. Prove that if U is a subspace of V not contained in W, then dim$W \bigcap U) =$ dim$U$ - 1.
\end{theorem}

\begin{proof}
	Let $\{w_{1},...,w_{n-1}\}$ be a basis for W, and $\{ u_{1},...,u_{k}\}$ a basis for U. Since U is not contained in W. there must be a vector in the basis of U that is linearly independent from the basis of W. Let us call this vector $v_{1}$. Then we extend the basis for W to a basis for V by $\{ w_{1},...,w_{n - 1},v_{1}\}$. We see that there can be at most one vector from the basis of U that is linear independent from the basis of W. $U = U/W + (U \bigcap W) \implies \dim{U} = \dim{U/W} + \dim{U \bigcap w}$ with $\dim{U/W} = 1$. 
\end{proof}

\begin{theorem}
	Let $f(x)$ be a polynomial of degree n. Prove that $\{f(x),f^{'}(x),...,f^{n}(x)\}$.
\end{theorem}

\begin{proof}
	The only function for which $f(x) = cf^{n}(x)$ is the function $f(x) = e^{x}$ which is not a polynomial. To see the linear indepence of the higher derivatives, we will use the fundemental theorem of Calculus. $f^{n}(x) = \int f^{n+1}(x)dx$\footnote{As always, this is true up to a constant}. So, if $f(x) \not = cf^{n}(x)$ then $f^{i}(x) \not = cf^{j}(x)$ for any $i , j$. So, the set is linearly indepedent.
\end{proof}

\begin{theorem}
	Let $a_{0},a_{1},...,a_{n}$ be scalars and $$p_{i}(x) \ \Pi_{k \not = i} \frac{x - a_{j}}{a_{i} - a_{j}}$$

	Prove that the set $\{ p_{1}(x), ..., p_{n}(x) \}$ is a basis for $\mathbb{P}_{n}$.
\end{theorem}

\begin{proof}

\end{proof}
\end{document}

