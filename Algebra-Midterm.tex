\documentclass{article}
\usepackage{amsmath}
\renewcommand{\baselinestretch}{1.05}
\newcommand{\norm}[1]{\left\lVert#1\right\rVert}
\usepackage{amsmath,amsthm,verbatim,amssymb,amsfonts,amscd, graphicx}
\usepackage[T1]{fontenc}
\usepackage{bigfoot} % to allow verbatim in footnote
\usepackage[numbered,framed]{matlab-prettifier}
\usepackage{filecontents}
\usepackage{graphics}
\usepackage{tikz-cd}
\topmargin0.0cm
\headheight0.0cm
\headsep0.0cm
\oddsidemargin0.0cm
\textheight23.0cm
\textwidth16.5cm
\footskip1.0cm
\theoremstyle{plain}
\newtheorem{theorem}{Theorem}
\newtheorem{corollary}{Corollary}
\newtheorem{lemma}{Lemma}
\newtheorem{proposition}{Proposition}
\newtheorem*{surfacecor}{Corollary 1}
\newtheorem{conjecture}{Conjecture} 
\newtheorem{question}{Question} 
\theoremstyle{definition}
\newtheorem{solution}{Solution}
\newtheorem{definition}{Definition}
\title{Analysis 1}
\author{Mike Desgrottes}
\date{August 2020}

\begin{document}
\maketitle
 \section{}
\begin{theorem}
Show that Set has a subobject classifier and determine if Grp has a subobject classifier. 
\end{theorem}

\begin{proof}
Let $ \omega = \{ 0,1\}$, we shall show that the diagram \begin{tikzcd}
X \arrow[r, "\phi"] \arrow[d, "j" red]
& 1 \arrow[d, "\psi" red] \\
A \arrow[r, red, "\eta" blue]
& \omega
\end{tikzcd} commutes for all monomorphism $j: X \to A$ where $1$ is a terminal object of Set. Now, $\phi$ is unique by virtue of 1 being terminal. If $\eta(x) = 1$ if $x \in X$, and $0$ otherwise, and $\psi(1) = 1$. Then necessarily, we see that $\eta(j(x)) = \psi(\phi)$. So, the diagram commutes. Suppose that $\Tilde{\eta}$ is another morphism that make the diagram commutes. Then $\eta(j(x)) = \Tilde{\eta}(j(x)) \implies \eta(x) = \Tilde{\eta}(x)$ whenever $x \in X$. This also implies that $\eta(x) = \Tilde{\eta}(x)$ for all x. The two morphisms coincide and are one and the same. 
\end{proof}
\begin{theorem}
Show that in every category with products/coproducts that $$ A \pi B \cong B \pi A$$ and $$ A \amalg B \cong B \amalg A $$
\end{theorem}

\begin{proof}
$A \pi B$, and $B \pi A$ are both final objects in the category $C_{A,B}$. $A \amalg B$, and $B \amalg A$ are both initial objects in the category $C^{A,B}$. Initial and final objects are isomorphic when they exists in the same category.
\end{proof}
\begin{theorem}
Let X be a set, A an abelian group and $\phi: X \to A$. Prove that there exists  a unique abelian group homomorphism $\Tilde{\phi} : G/H \to A$ such that 
\begin{tikzcd}
X \arrow[rd, "\phi"] \arrow[r] & G/H \arrow[d, "\Tilde{\phi}"] \\
& A
\end{tikzcd} commutes where $G = F(x)$ and $H = [G,G]$

\end{theorem}

\begin{proof}
We shall show that the following diagram commutes. 

\begin{tikzcd}
X \arrow[rd, "\phi"] \arrow[r, "j"] & G \arrow[r, "\psi"] \arrow[d, "\Tilde{\psi}"]& G/H \arrow[ld, "\Tilde{\phi}"]\\
& A
\end{tikzcd}

G being a free group forces $\Tilde{\psi}$ to be unique and the first triangle commutes. The first isomorphism theorem makes the second triangle commutes. Therefore, the outer triangle also commute because $\Tilde{\psi}(\psi(j(x))) = \phi(x)$. The uniqueness of $\Tilde{\phi}$ implies that $G/H$ is an initial object in the category $F^{A}$. $F^{Ab}(X)$ is also an initial object. Therefore $G/H \cong F^{Ab}(X)$.

\end{proof}

\begin{theorem}
Let $\phi: G \to H$, and $\psi: H \to K$ are morphisms in a category with products. Prove that $$ (\psi \circ \phi) \pi (\psi \circ \phi) = ( \psi \pi \psi) \circ (\phi \pi \phi)$$
\end{theorem}

\begin{proof}

\end{proof}
\begin{theorem}
Let $H \leq G$ be a subgroup. Prove that 

\begin{tikzcd}
G \times G/H \arrow[r] & G/H \\ 
\end{tikzcd} and 

\begin{tikzcd}
G \times G/(gHg^{-1}) \arrow[r] & G/(gHg^{-1}) \\
\end{tikzcd} are 3isomorphic
\end{theorem}/   
\begin{proof}
H is normal. $H = gHg^{-1}$ for all $g \in G$. 
\end{proof}

\begin{theorem}
Let $G_{1},G_{2}$ be two groups and $\phi: F(G_{1}) \to G_{1}$, and $\psi: F(G_{2}) \to G_{2}$ be natural epimorphism. Prove that $G = F(G_{1} \bigcup G_{2})/< Ker \phi_{1}, Ker\phi_{2}>$ is a coproduct in Grp.
\end{theorem}

\begin{proof}
First, we see that G come with pair of morphism. $j_{G_{1}}: g_{1} \to [g_{1}]$ and $j_{G_{2}}: g_{2} \to [g_{2}]$ by sending each element of the group to their equivalence classes in the quotient group. If we let $Z$ be  a group endowed with pair of morphism $f_{1}, f_{2}$, then we have to show that there exists a unique morphism $\sigma$ such that the diagram \begin{tikzcd}
G_{1} \arrow[d, "j_{G_{1}}"] \arrow[rd, "f_{1}"] \\
G \arrow[r, "\sigma"] & Z  \\
& G_{2} \arrow[lu, "j_{G_{2}}"] \arrow[u, "f_{2}"] \\
\end{tikzcd}. The map given by $\sigma([g_{1}]) = f_{1}(g_{1})$ and $\sigma([g_{2}]) = f_{2}(g_{2})$ where $g_{1} \in G_{1}$ and $g_{2} \in G_{2}$
\end{proof}

\begin{theorem}
Prove that Grp has cokernels. Determine if Grp has coequalizer. 
\end{theorem}

\begin{proof}
	Let $\phi: G \to G^{'}$, we proceed by proving that $coker \phi = G^{'}/N$ where N is the smallest normal subgroup that contains $im \phi$. Intersection of normal subgroups is again normal. 
\begin{tikzcd}
	G \arrow[r, "\phi"] & G^{'} \arrow[r, "\alpha"] \arrow[d, , "\phi"]  & L  \\
			    & G^{'}/N \arrow[ur, "\psi"] \\
\end{tikzcd}

The diagram commute with $\psi(gN) = \alpha(g)$ and $\phi(g) = gN$. The function is well-defined because N is normal. $gN = hN \implies gh^{-1} \in N \implies \psi(gN) = \psi(hN) $ because $\psi(gh^{-1}N) = \psi(N)$. The uniqueness of $\psi$ is because it is completely determined by $\alpha$.
\end{proof}
\end{document}

