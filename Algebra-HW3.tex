\documentclass{article}
\usepackage{amsmath}
\renewcommand{\baselinestretch}{1.05}
\newcommand{\norm}[1]{\left\lVert#1\right\rVert}
\usepackage{amsmath,amsthm,verbatim,amssymb,amsfonts,amscd, graphicx}
\usepackage[T1]{fontenc}
\usepackage{bigfoot} % to allow verbatim in footnote
\usepackage[numbered,framed]{matlab-prettifier}
\usepackage{filecontents}
\usepackage{graphics}
\topmargin0.0cm
\headheight0.0cm
\headsep0.0cm
\oddsidemargin0.0cm
\textheight23.0cm
\textwidth16.5cm
\footskip1.0cm
\theoremstyle{plain}
\newtheorem{theorem}{Theorem}
\newtheorem{corollary}{Corollary}
\newtheorem{lemma}{Lemma}
\newtheorem{proposition}{Proposition}
\newtheorem*{surfacecor}{Corollary 1}
\newtheorem{conjecture}{Conjecture} 
\newtheorem{question}{Question} 
\theoremstyle{definition}
\newtheorem{definition}{Definition}
\title{Analysis 1}
\author{Mike Desgrottes}
\date{October 2020}

\begin{document}
\maketitle

\begin{theorem}
	Suppose that $\alpha \in End(v)$ and $V = W_{1}  \bigoplus .. \bigoplus W_{k}$ such that each subspace is $\alpha$-invariant. Prove that there exists a basis $B$ for $V$ such that the representation matrix of $\alpha$ with resprct to B has the form $A = [A_{ij}]$ are zero matrices for all $i \not = j$.


\end{theorem}

\begin{proof}
	Let $B_{i}$ be a basis for $W_{i}$, then the matrix representation of $\alpha$ with respect to $B_{i}$ is contained in $W_{i}$ because $W_{i}$ is $\alpha$-invariant. $\bigcup_{i = 1}^{k} B_{i}$ is a basis for V. Then for every $v \in B_{i}$, then $\alpha(v) \not \in B_{j}$ for $j \not = i$. Hence $A = diag(A_{11},...,A_{kk})$.
\end{proof}

\begin{theorem}
	Let A be a matrix such that its characteristic polynomial is completely reducible. Prove that A is nilpotent if and only if $spec(\alpha) = \{0\}$.
\end{theorem}

\begin{proof}
	Since the characteristic polynomial of A is completely reducible, then A is similar to a diagonal matrix. Hence, the diagonal matrix is also nilpotent. Let $B = diag(\lambda_{1},...,\lambda_{n})$ is the diagonal matrix, then $B^{k} = diag(\lambda_{1}^{k},...,\lambda_{n}^{k}) = 0 \implies \lambda_{i} = 0$ for all i. So, A being nilpotent implies $spec(A) = \{0\}$.

	Suppose that the spectrum of A is trivial, then A is similar to the zero matrix. Hence $A = P^{T}BP = 0$ which is nilpotent. 
\end{proof}

\begin{theorem}
	Prove that if J is a m x m Jordan block with zero maind diagonal, then $det(tI - J) = t^{m}$ is the minimal polynomial of  J.
\end{theorem}
\begin{theorem}
	let $p(t) = t^{3} - 6t^{2} + 11t - 6$. Describe the Jordan Canonical forms of the 3 x 3 matrices A that statisfies $p(A) = 0$ if such matrices exist.
\end{theorem}
\end{document}

